\documentclass[a4paper,11pt]{article}
\usepackage[utf8]{inputenc}
\usepackage[T1]{fontenc}
\usepackage[frenchb]{babel}

\usepackage{graphicx}
\usepackage{fancyhdr}
\usepackage{geometry}

\usepackage[colorlinks,linkcolor=blue]{hyperref}
\usepackage{amsmath}
\usepackage{amssymb}
\usepackage{mathrsfs}
\usepackage{epsfig}
\usepackage {eurosym}

\usepackage{float}

\geometry{a4paper,tmargin=2cm,bmargin=2cm,lmargin=1.5cm,rmargin=1cm,headheight=2.2cm,headsep=0.5cm,footskip=1cm}
\columnsep=0.6cm

\graphicspath{{images/}} 

\usepackage{listings}
\usepackage{color}
\usepackage{xcolor}

\lstset{columns=flexible,keepspaces=true, breaklines,breakindent=0pt} 


\lstset{language=VHDL,
basicstyle=\ttfamily\footnotesize,
breaklines, 
keywordstyle=\bfseries\color{blue},
stringstyle=\color{red},
commentstyle=\color{blue!20!black!30!green},
morecomment=[s][\color{black}]{/**}{*/},
numbers=left,
numberstyle=\tiny\color{black},
stepnumber=2,
numbersep=10pt,
tabsize=4,
showspaces=false,
showstringspaces=false}


\fancypagestyle{plain}{
% noms des respo   dans le bas de page                                            
\lfoot{Projet de VHDL}
\rfoot{M.Morin J.Fourmann}
\renewcommand{\headrulewidth}{0pt}
\fancyhead{}}
% Titre a compl»ter
\title{\textbf{ \huge{Projet de VHDL}}  \\{\Large Réalisation d'un Fréquencemètre}}

\author{
\textsc{Jérémie Fourmann} (Promo 2013 - Eléctronique)\\ %mettre votre nom
\textsc{Maxime Morin} (Promo 2013 - Eléctronique)\\ %mettre votre nom
%\textsc{ddd dddd} (Promo - departement - respo)     %2 nom
}

\graphicspath{{images/}}

\begin{document}

\pagestyle{plain}

\maketitle
\begin{center}
\includegraphics[width=6cm]{inp-enseeiht.pdf}   
\end{center}

\vspace{1cm}
\renewcommand{\contentsname}{Plan}
\tableofcontents
\vspace{2cm}

\newpage
\section{Objectifs}
 
\subsection{Rappel du cahier des charges}
\subsection{Nos choix}

\newpage
\section{Conception du système}
\subsection{Présentation du système}
\subsection{Présentation des sous Modules}
\subsubsection{Schéma bloc}
\subsubsection{Machine d'état}
\subsubsection{Résulat de simulation}

\subsection{Performance du système}

\newpage
\section{Bilan}

\newpage
\appendix 
\section{Manuel d'utilisateur}

\newpage
\appendix 
\section{Extrait de code VHDL}
\lstinputlisting{../machine_etat.vhd}
\end{document}
